\documentclass{article}
\usepackage[utf8]{inputenc}

\usepackage{amsmath,amssymb,amsthm}

\title{Gossip Problem}

\newtheorem{lemma}{Lemma}
\newtheorem{theor}{Theorem}

\begin{document}
\paragraph{The gossip problem.\\}
There are $n$ people, each with a information, and they want to share all the infos in the least number of calls, given that two people calling each other share all the infos that they have.\\
Let's think about this problem as a labeled graph, with vertexes representing people, edges calls, and labels natural numbers which indicates the order of the calls.\\
Such a graph $G$ is called a solution if sharing the information between the couples of vertexes in the order given by the labels yields the total information to each vertex.
\paragraph{Some preliminary bound\\}
To step into the problem let's observe two obvious bounds.\\
Let $S(n)$ be the minimal number of calls needed to solve $n$; then
\begin{equation*}
\frac{3}{2}n-2\leq S(n)\leq 2n-3
\end{equation*}
The lower bound is given by the fact that at least $n-1$ calls are needed in order for at least $1$ person (actually $2$) knows everything, and then every other $n-2$ people has to do at least one call. The upper bound is given by the obvious solution where everybody call one person who then calls everybody else back.
\paragraph{A better bound and the solution\\}
We can improve the upper bound.\\
A solution for $n=4$ is given by the circular $4-$graph labeled $(1,3,2,4)$ following any closed path, namely a solution where no two consecutive calls involve a common person.\\
This solution takes $4=2*4-4$ steps and can be extended to any $n$ adding $2(n-4)$ calls, namely every added person call one of those in the square, they apply their solution and then inform the others back.\\
We claim that this solution, even though is not the unique, satisfy the \emph{optimal} upper bound \begin{equation*}
S(n)\leq 2n-4
\end{equation*}
\paragraph{Making the calls\\}
We need some preliminary lemmas.\\
Adding people doesn't help.
\begin{lemma}
Lets denote $S(n,m)$, with $n\leq m$, the optimal solution for $n$ people among $m$, then
\begin{equation*}
S(n,m)=S(n)
\end{equation*}
\end{lemma}
This is because we could just call the person who introduced the new guy instead.\\
Time is reversible.
\begin{lemma}
Let $(G,(e_1,...,e_k))$ be a solution for $n$, then $(G,(e_k,..,e_1))$ is a solution for $n$.
\end{lemma}
To prove it just observe that there is a symmetry between knowing everything and having spread one's information everywhere.\\
Surprised people like to talk.
\begin{lemma}
Suppose $e_k$ is the last call of an optimal solution that involves vertexes $A,B$, and suppose that they both learn something with this last call, then there are at least $3$ calls connecting $\{A,B\}$ to the rest of the vertexes.
\end{lemma}
Here the point is that if we make our own call too early we couldn't discover everything, while if we make it too late, then the rest of the group must spend some time to talk about our gossip.\\
Stranges are the same.
\begin{lemma}
If two following calls (or blocks of calls) with no common vertex appear in a solution $G=(...,e_m,e_{m+1},...)$, then switching those calls still gives an equivalent solution $G'=(...,e_{m+1},e_m,...)$.
\end{lemma}
From which follows that...\\
Climax is important.
\begin{lemma}
Any solution $G$ admits an equivalent solution $G'$ such that the calls where somebody learns everything are all at the end.
\end{lemma}
Now we can prove our theorem.
\begin{theor}
The optimal solution for $n\geq 4$ is $S(n)=2n-4$.
\end{theor}
The proof goes by induction, with base step $n=4$ which has already been proved.\\
Now we are in the inductive step with $n$ people, and we suppose by contradiction that $S(n)\leq 2n-5$.\\
Lets consider the last call $e$ of such a solution $G$, and call $A$ and $B$ the vertex involved: either one of them, say $A$, already new everything (and is informing the other), either they both learn something new.\\
In the first case we can produce a solution $S(n-1)\leq 2n-4$ by eliminating $B$.\\
If both learn something then the second last call cannot have involved one of them, hence we have an equivalent solution permuting the last and the second last call, and we have the additional information that these have no vertex in common.\\
Applying inductively the last two observation we find that, as long as someone knows everything, the $l-$th last call must involve $2$ peoples which know everything after that call, but learn something during it.\\
Thus, if $n$ is odd, we find a contradiction as in the first observation.\\
If $n$ is even we know that the last $n/2$ calls have no pairwise common vertexes, and by Lemma 2, there exist a solution such that the same is true for the first two $n/2$ calls as well.\\
Then we have $n$ vertexes which are coupled by $n/2$ calls at the beginning and at the end, furthermore no couple is the same at the beginning and at the end, by Lemma 3.\\
We can chose one vertex from any initial couple, let them solve the $n/2$ problem in $S(n/2)$ steps, and then call their ending coupled vertex. Any different solution without such a symmetry can be modified without changing the number of calls to have this symmetry, redirecting the calls which involves vertex that has not been chosen.\\
Hence we either forced our initial solution to be $n/2+S(n/2)+n/2$ long, either to have $S(n/2)\leq n-4$, which is anyway a contradiction.\\\qed\\
As a final observation we may note that the proof actually give us plenty of ways to construct optimal solutions from smaller ones.
\paragraph*{Variations\\}
We may ask for variations of this problem.\\
One straightforward generalization would we to allow only some call to be performed, i.e. bounding the solution to live in a fixed non-complete graph.\\
It is immediate to check that the optimal bound for the solution is still reached, as long as the graph contain a 4-cycle. This problem shows then a remarkable stability in this sense.
\end{document}
